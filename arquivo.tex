\documentclass[12pt,a4paper]{article}
\usepackage[utf8]{inputenc}
\usepackage[brazilian]{babel}
\usepackage{amsmath}
\usepackage{amsfonts}
\usepackage{amssymb}
\usepackage[T1]{fontenc}
\usepackage{graphicx}
\usepackage{physics}
\usepackage{siunitx}
\newcounter{prob}
\newcounter{subprob}
\renewcommand{\thesubprob}{\alph{subprob}}

\newcommand{\problem}{\setcounter{subprob}{0} \stepcounter{prob} \par \medskip \noindent \textbf{Problema~\theprob \ }}

\newcommand{\answer}{\par \medskip \noindent \textit{Resposta \ }}

\newcommand{\finalanswer}[1]{
	\begin{center} 
    	{\renewcommand{\arraystretch}{1.5}
		\renewcommand{\tabcolsep}{0.2cm} 
    	\begin{tabular}{|c|} 
    		\hline 
        	$ \displaystyle #1 $  \\ 
        	\hline 
    	\end{tabular}} 
   	\end{center}}

\newcommand{\subproblem}{\stepcounter{subprob} \par \smallskip \noindent \quad \textit{(\thesubprob) \ }}

\newcommand{\subanswer}{\par \smallskip \noindent \quad \textit{Resposta \ }}

\newcommand{\option}{\item[$\square$]}
\newcommand{\thisone}{\item[$\blacksquare$]}

\newenvironment{subitemize}{\begin{itemize}}{\end{itemize}}


\author{Aluno - Anderson Junior Novacoski}
\title{Lista de Exercícios - Teoria da Computação}
\date{}
\begin{document}

%%--CABEÇALHO--%%
	\begin{center}
    {\huge Lista de Exercícios \par}
    {\LARGE Disciplina \par}
    {\Large Teoria da Computação \par}
	\end{center}

\problem Procurar outros exemplos de instâncias de PCPs que têm solução.
\answer{
  \begin{table}[h]
  \centering
    \begin{tabular}{|r|r|r|r|r|r|}
        \hline
        \textbf{v1}&\textbf{v2}&\textbf{v3}&\textbf{v4}&\textbf{v5}&\textbf{v6}\\
        \hline
        01&00&001&10&100&11\\
        \hline
    \end{tabular}
  \end{table}
}
\finalanswer{\text{Esta é a resposta final ao problema 1}}

\problem Procurar exemplos de instâncias de PCPs que não têm solução e justificar.
\answer Este é o desenvolvimento da resposta ao problema 1, incluindo os passos lógicos necessários e etc...
\finalanswer{\text{Esta é a resposta final ao problema 2}}

\problem Encontre a solução (se existir) para cada um dos sistemas de correspondência de Post:
	\subproblem (0, 10, 0); B = (10, 1, 01)
        \subanswer{
                  \begin{table}[h]
                  \centering
                    \begin{tabular}{|l|l|l|}
                        \hline
                        \textbf{v1}&\textbf{v2}&\textbf{v3}\\
                        \hline
                        0&10&0\\
                        \hline
                        10&1&01\\
                        \hline
                    \end{tabular}
                  \end{table}
                  \\
                  \textbf{Não é possível resolver.} Como só é possível começar com a sequência V2, ou V3(cuja parte de cima começa com os mesmos dígitos da parte de baixo)então não há como seguir adiante, afinal teríamos:\\
                  v2->v3 = $\frac{100}{101}$\\
                  v2->v1 = $\frac{100}{110}$\\
                  v2->v2 = $\frac{1010}{11}$\\

                  v3->v1 = $\frac{00}{0110}$\\
                  v3->v2 = $\frac{010}{011}$\\
                  v3->v3 = $\frac{00}{0101}$\\
                  Gerando uma parada sem obter resposta.

        }
	\subproblem (01, 001, 10); B = (011, 01, 00)
	\subanswer Subrespostas também são possíveis
	\subproblem (01, 001, 10); B = (011, 10, 00)
	\subanswer Subrespostas também são possíveis
\problem Apesar do PCP ser indecidível, é fácil alterar o problema para que se torne decidível. Pense em um algoritmo de decisão para o PCP num sistema de correspondência de Post composto por um alfabeto de um único símbolo
\answer Este é o desenvolvimento da resposta ao problema 1, incluindo os passos lógicos necessários e etc...
\end{document}

